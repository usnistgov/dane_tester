%\documentclass[review]{elsarticle}
\documentclass[preprint,5p]{elsarticle}

\usepackage{lineno,hyperref}
\modulolinenumbers[5]

\journal{Journal of \LaTeX\ Templates}

%%%%%%%%%%%%%%%%%%%%%%%
%% Elsevier bibliography styles
%%%%%%%%%%%%%%%%%%%%%%%
%% To change the style, put a % in front of the second line of the current style and
%% remove the % from the second line of the style you would like to use.
%%%%%%%%%%%%%%%%%%%%%%%

%% Numbered
%\bibliographystyle{model1-num-names}

%% Numbered without titles
%\bibliographystyle{model1a-num-names}

%% Harvard
%\bibliographystyle{model2-names.bst}\biboptions{authoryear}

%% Vancouver numbered
%\usepackage{numcompress}\bibliographystyle{model3-num-names}

%% Vancouver name/year
%\usepackage{numcompress}\bibliographystyle{model4-names}\biboptions{authoryear}

%% APA style
%\bibliographystyle{model5-names}\biboptions{authoryear}

%% AMA style
%\usepackage{numcompress}\bibliographystyle{model6-num-names}

%% `Elsevier LaTeX' style
\bibliographystyle{elsarticle-num}
%%%%%%%%%%%%%%%%%%%%%%%

\begin{document}

\begin{frontmatter}

\title{Implementing DANE Email Testers}
\tnotetext[mytitlenote]{Fully documented templates are available in the elsarticle package on \href{http://www.ctan.org/tex-archive/macros/latex/contrib/elsarticle}{CTAN}.}

%% Group authors per affiliation:
\author{Simson Garfinkel}

\begin{abstract}
DNS-Based Authentication of Named Entities (DANE) can be used to
improve the trustworthiness of email by providing a discovery
mechanism for certificates and public keys that are used to secure the  SMTP communications
between Mail Transfer Agents (MTAs) and to encrypt email. A testing
frameworks for each of these DANE applications was created. The SMTP
tester uses a conventional logic-driven approach for test
frameworks, while the OPENPGP and SMIMEA testers employ a
database-driven testing framework that implements the tester as a
state machine. The two approaches are compared in terms of
development time, functionality, and extensibility.
\end{abstract}

\begin{keyword}
DNS-Based Authentication of Named Entities\sep DANE\sep OPENPGP\sep SMIMEA\sep 
\end{keyword}

\end{frontmatter}

\linenumbers

\section{Introduction}

Public key discovery and certification has been one of the primary
barriers to realizing the benefit of email security
protocols. 

Although many SMTP servers now implement the STARTTLS SMTP
command\cite{rfc3207}, Mail Transfer Agents (MTAs) typically
certificates signed by \emph{any} Certificate Authority (CA)
(including self-signed certificates), as SMTP servers generally lack a
list of pre-configured certificate authorities and a user interface
that could be used to specify the disposition of mail that is being
sent to a mail server that has a certificate that is signed by an
unknown CA. Since mail is sent by default without encryption, this
so-called ``opportunitist encryption'' at least provides security
against a passive monitoring attacker.

After more than two decades of research and standards development
(e.g. \cite{rfc1421,rfc4880,rfc5750}), two incompatiable standards
have emerged for the exchange of encrypted Internet email: S/MIME and
PGP. Both of these standards from the \emph{key discovery problem:} in
order to send a recipient an encrypted message, it is necessary to
first obtain the recipient's public key. Within a single organization,
this problem can be solved through the use of a enterprise
directory. S/MIME also has the advantage that digitally signed email
messages typically include the sender's public key in the PKCS\#7
attachment\cite{rfc2315}, allowing the recipient of a digitally signed
message to send an encrypted reply to the sender. However, when
sending email between organizations, there is currently no deployed
mechanism that can be used to find the recipient's S/MIME certificate
or PGP public key prior to the first message exchange.

DNS-Based Authentication of Named
Entities\cite{rfc7671} (DANE) provides a general mechanism for the
discovery of cryptographic keys associated with DNS-named. By relying
on the DNSSEC\cite{rfc3833} trust roots, DANE delegates control of the
names within an organization's DNS hierarchy to the administrative
unit that is in control of the organization's DNS. RFC7671 and RFC7672 provide a
discovery mechanism for a domain's certified SMTP MTA public key or
PEM certificate\cite{rfc7671,rfc7672}. Work is in progress on a mechanism for discovering
S/MIME certificates associated with email addresses
(\texttt{draft-ietf-dane-smime11}), and 
a mechanism for associating OpenPGP public keys with email addresses (\texttt{draft-ietf-dane-openpgpkey-12}).

We created a conventional tester for HTTP and SMTP servers in Python. The tester consists of
a python module that implements an instrumented client that uses
getdns to acquire a TLSA RR for either an HTTP or SMTP server. The
certificate is then used to verify a TLS server reached through either
HTTP or SMTP. The tester can be run from command line or a web
interface. Testing results are incorporated into a data structure
which can then be rendered as either text or HTML.  

The conventional tester was implemented using commonly accepted programming
practices including object-oriented design, modularization, and
integrated tests. Nevertheless, it proved difficult to
develop, audit, and extend to new protocols. A particular complication
is the fact that whether a DANE validation succeeds or fails depends
on both the server's certificate and the results of DNS queries, but DNS servers frequently give
different answers to the same questions as a result of caching, load balancing, network
congestion, and configuration changes. Thus, it is difficult to
determine the reason for code that suddenly starts or stops working.

Following the development of the conventional tester, we created a
tester for the draft SMIMEA and OpenPGP protocols. As a result of the
experience with the conventional HTTP and SMTP testers, we developed a
new tester framework based on the core idea of \emph{auditability},
which we define as ``the degree to which transactions can be traced
and audited through a system.'' The core design goal of the SMIME and
OpenPGP tester was to maintain a log of all DNS queries and responses
so that the tester's edicts could be reviewed and validated. Another
design goal was to enable a \emph{replay} capability, so that previous
queries and responses to be used for regression testing. In the
process of designing the new testing framework, we realized the entire
testing agent could be implemented as a state machine, with each phase
of the test corresponding to states and state transitions being
directed by the responses from the systems undergoing test.





\section*{References}

\bibliography{rfc}


\end{document}

% LocalWords:  cryptographic modularization auditability
